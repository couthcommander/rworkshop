\documentclass[twocolumn]{article}

\usepackage{framed}
\usepackage{gillius2}
\usepackage{nopageno}
\usepackage[margin=0.75in]{geometry}
\setlength{\columnseprule}{0.15pt}
\setlength{\parskip}{2mm}

\renewcommand{\familydefault}{\sfdefault}

\begin{document}

\textbf{\LARGE{\textsc{R \& RStudio Workshop}}}

\center{Monday, May 10\textsuperscript{th} 2021, 9:00am--4:00pm}

\vspace*{\fill}

\begin{flushleft}

\textbf{Instructor:}

Cole Beck, BS

\texttt{email:cole.beck@vumc.org}

Principal Application Developer

Vanderbilt University Medical Center
\end{flushleft}

\vspace*{\fill}

\begin{flushleft}
\textbf{Location:}

This is an online presentation. A Zoom link will be provided.
\end{flushleft}

\vspace*{\fill}

\begin{flushleft}

\textbf{Requirements:}

Rudimentary Knowledge of Programming

Laptop with R, and RStudio installed.

Please follow these instructions \textit{before} the class begins. This must be emphasized, as we are unable to help install or configure your system during the presentation. If these are not installed and ready to use, you will not be able to follow along interactively. If you need help, please reach out to your IT support, or the author (email above) before May 10.

\end{flushleft}

\vspace*{\fill}

\begin{flushleft}

\textbf{Installing R}

Follow the instructions on \texttt{https://mirrors.nics.utk.edu/cran/}.

\textbf{Install RStudio}

Find the appropriate installer for your platform at \texttt{https://www.rstudio.com/}. Free installers are toward the bottom of the download page.

\textbf{R Packages}

Install the packages \texttt{Hmisc}, \texttt{rms}, \texttt{knitr}, \texttt{rmarkdown}, \texttt{plotly}, and \texttt{devtools} using RStudio and any pending upgrades.

From RStudio, \texttt{Tools $\rightarrow$ Install Packages}

Or alternatively from the R command line do the following.

\begin{framed}
\begin{verbatim}
install.packages(c('Hmisc', 'rms', 'knitr',
  'rmarkdown', 'plotly', 'devtools'))
  
update.packages(checkBuilt=TRUE, ask=FALSE)
\end{verbatim}
\end{framed}

\end{flushleft}

\vspace*{\fill}

\pagebreak

\begin{flushleft}

\vspace*{0.5cm}

\textbf{Schedule:}

\begin{tabular}[H]{lll}
8:00 (CST) & 9:00 & Zoom Meeting Opens \\
9:00 & 10:00  & Class Begins w/ Foundations \\
10:15 & 11:00 & Plotting \\
11:00 & 12:00 & Models and Data Manipulation \\
12:00 & 1:00  & Lunch \\
1:00  & 2:00  & Introduction to RMS (OLS) \\
2:15  & 3:00  & Cleaning Data \\
3:00  & 4:00  & Simulation \\
\end{tabular}

\vspace*{\fill}

\textbf{Course Materials}

Course materials were created by Cole Beck and available online at
\texttt{https://github.com/couthcommander/rworkshop}. The answers to
all exercises are posted as well. It is \emph{not} recommended to use
these until one has at least attempted the exercises. There is an intro document that will not be covered in the course and is a basic guide to programming in R. This is a good quick review of programming in R.


\vspace*{\fill}

\textbf{Recommended Additional Materials}

RStudio has a variety of helpful ``cheatsheets'' available at \texttt{rstudio.com/resources/cheatsheets}. We recommend the following if desired:

\begin{itemize}
\item RStudio IDE
\item R Markdown
\item Base R
\end{itemize}

\vspace*{\fill}

Day 1 of Regression Modeling Strategies Short Course by Frank Harrell. Course includes applications to Linear Models, Logistic and Ordinal Regression and Survival Analysis.

\end{flushleft}

\end{document}
