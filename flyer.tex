\documentclass[twocolumn]{article}

\usepackage{framed}
\usepackage{gillius2}
\usepackage{nopageno}
\usepackage[margin=0.75in]{geometry}
\setlength{\columnseprule}{0.15pt}
\setlength{\parskip}{2mm}

\renewcommand{\familydefault}{\sfdefault}

\begin{document}

\textbf{\LARGE{\textsc{R \& RStudio Workshop}}}

\center{Monday, May 13\textsuperscript{th} 2019, 9:00am--4:00pm}

\vspace*{\fill}

\begin{flushleft}

\textbf{Instructor:}

Shawn Garbett, MS

\texttt{email:Shawn.Garbett@vumc.org}

Sr. Application Developer

Vanderbilt University Medical Center
\end{flushleft}

\vspace*{\fill}

\begin{flushleft}
\textbf{Location:}

Vanderbilt University Alumni Hall

2205 West End Avenue, Rooms 201 and 202

Nashville, TN 37203
\end{flushleft}

\vspace*{\fill}

\begin{flushleft}

\textbf{Requirements:}

Rudimentary Knowledge of Programming

Laptop with R, and RStudio installed.

\end{flushleft}

\vspace*{\fill}

\begin{flushleft}

\textbf{Installing R}

Follow the instructions on \texttt{https://mirrors.nics.utk.edu/cran/}.

\textbf{Install RStudio}

Find the appropriate installer for your platform at \texttt{https://www.rstudio.com/}. Free installers are toward the bottom of the download page.

\textbf{R Packages}

Install the packages \texttt{Hmisc}, \texttt{rms}, \texttt{knitr}, and \texttt{rmarkdown} using RStudio and any pending upgrades.

From RStudio, \texttt{Tools $\rightarrow$ Install Packages}

From the command line,

\begin{framed}
\begin{verbatim}
install.packages(c('Hmisc', 'rms', 'knitr',
  'rmarkdown', 'plotly'))
  
update.packages(checkBuilt=TRUE, ask=FALSE)
\end{verbatim}
\end{framed}

\end{flushleft}

\vspace*{\fill}

\pagebreak

\begin{flushleft}

\vspace*{0.5cm}

\textbf{Schedule:}

\begin{tabular}[H]{lll}
9:00 & 10:00  & Foundations \\
10:15 & 11:00 & Plotting \\
11:00 & 12:00 & Models and Data Manipulation \\
12:00 & 1:00  & Lunch \\
1:00  & 2:00  & Introduction to RMS (OLS) \\
2:15  & 3:00  & Cleaning Data \\
3:00  & 4:00  & Simulation \\
\end{tabular}

\vspace*{\fill}

\textbf{Course Materials}

Course materials were created by Cole Beck and available online at
\texttt{https://github.com/couthcommander/rworkshop}. The answers to
all exercises are posted as well. It is \emph{not} recommended to use
these until one has at least attempted the exercises. There is an intro document that will not be covered in the course and is a basic guide to programming in R. This is a good quick review of programming in R.


\vspace*{\fill}

\textbf{Recommended Additional Materials}

RStudio has a variety of helpful ``cheatsheets'' available at \texttt{rstudio.com/resources/cheatsheets}. We recommend the following if desired:

\begin{itemize}
\item RStudio IDE
\item R Markdown
\item Base R
\end{itemize}

\vspace*{\fill}



\end{flushleft}

\end{document}